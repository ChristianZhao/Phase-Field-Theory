% !TEX root = Clean-Thesis.tex
%
\pdfbookmark[0]{Abstract}{Abstract}
\chapter*{Abstract}
\label{sec:abstract}
\vspace*{10mm}


\textit{\large
Unique nature of disordered structure in Metallic Glasses provides the combination of exceptional physical properties like at one end they are among the hardest materials ever made while on the other end pretty easier to work and form despite their hardness. Due to the lack of long range periodicity and related grain boundaries with defects, Metallic Glasses are reaching the limits of the theoretical strength. Their net shape processability by Cu-mold casting, suction casting and super plastic forming in Newtonian viscous flow regime (supercooled region) have led these materials to replicate in small features and in thin sections with high aspect ratios. Also, Metallic Glasses do not require post-cast processing or heat treatment, as all properties are already achieved in as-cast state. Also they could be fabricated and processed at lower temperatures comparative to their crystalline counterparts. Almost zero shrinkage during fabrication for these alloys makes them perfect for precision casting. These ideal properties have made possible their use in a number of fields ranging from very sophisticated tools like micro-geared motors to major commercial applications in the present world. The major applications of these precious metals are in sports items, hard casings, aircrafts, surgical instruments, industrial coatings and energy savings as amorphous core transformers and in various kind of sensors i.e. pressure sensor in auto mobile industry for injection control, oil pressure control, air conditioning, clogging monitor, brake control, and magnetic and magneto-optical sensors i.e. current sensation and optical communication.
}
